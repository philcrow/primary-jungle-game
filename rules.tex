\documentclass[twocolumn]{article}
\begin{document}
\title{Primary Jungle or How to Train Your Candidate}
\author{Phil Crow}
\maketitle

\section{Object}

You are a highly sought after campaign manager who could work for any candidate. Choose a candidate to guide to a first place finish in
the Iowa Caucuses. In this parallel universe, those are juggle caucuses where all candidates run together, regardless of party or ideology.

\section{Supplies}

\begin{itemize}
\item Candidate cards
\item Board
\item Calendar piece (small blue cube)
\item 50 White 1 point poll chips, 10 Red 5 point poll chips
\item Two white dice and one blue die
\item Money, divided into Super PAC and Candidate
\item 137 Weapon cards\footnote{You might be wondering, ``Why not 140?'' The other three were honesty, integrity, and foresight. Think of which candidates could genuinely play these. }
\item Field office pieces (small plastic stars one color per player)
\end{itemize}

\section{Starting the Game}

\subsection{Choose a Candidate}

Each player chooses one candidate card. Fight amongst yourselves (euphemistically please). Each candidate has a power.
Pay attention to that power and follow its instructions whenever they apply. For instance, now is a good time for Earnie Painter to open that
free New Hampshire field office by placing a star in New Hampshire on the board, if at least three people are playing.

\subsection{Start the Calendar}

Place the calendar piece on the board.  Starting calendar squares are marked with a line for two players, a triangle for three, etc. If you like formulas,
multiply the number of players by 15. Add one.

\subsection{Divide the Voters}

Place 15 poll chips on the undecided area of the board. Divide the remaining chips equally among the players. Place any remainder on undecided.

\subsection{Raise PAC Cash}

Each player rolls the two white dice and collects that total amount times \$100,000 in Super PAC bills. This is your candidate's PAC money. You can spend it on Warm Fuzzy Ads or
to take a risk with Attack Ads. Ronald Chump does not use PAC money, so skip this step, if he is your candidate. Jethro Shrub possesses the golden rolodex and gets double his roll.

\subsection{Raise Initial Candidate Cash}

Each player receives \$10K in candidate seed money for each opponent faced. Example: If there are five players, each player receives \$40K in candidate cash.
Note that Ronald Chump has unlimited funds. The bank covers for all of his spending. Skip this step, if he is your candidate. Melodie Whiteman and Fiona Carlson
have the power to raise extra money in this step. Follow their power instructions, if one of them is your candidate.

\subsection{Shuffle up and Deal}

Shuffle the weapon cards. Deal five to each player. Place the rest on the weapons draw pile on the board.

\subsection{Choose a First Player}

Determine who will go first in any way you like. If you can't think of a way, go in height order from tallest to shortest. Height is a great advantage in politics.

\section{Turn Phases}

Each turn has the following phases in order. Details on each phase are below.

\begin{enumerate}
\item Move the calendar piece
\item Raise money
\item Pay for staff
\item Play a weapon card (optional)
\item Open or close field offices (optional)
\item Draw (if you played a weapon)
\end{enumerate}

\subsection{Move the Calendar}
Every turns begins by moving the calendar forward one ``day.'' Pay attention so caucus day does not sneak up on you.

\subsection{Raise Money}
Ronald Chump skips this step.

Everyone participating in this step raises at least \$5k regardless of their roll.
To see if you raise more than \$5k, throw all three dice.
Pay attention to endorsement cards and Mallory Hinton's power when calculating totals.

If the blue die is three or less: subtract the total of the white dice from your current poll chip count. This is your final total.

If the blue die is four or more: add the total of the white dice to your current poll chip count. This is your final total.

In both cases, take your final total from the bank in thousands using candidate cash bills, but not less than \$5k. 

\subsection{Pay for Staff}
Ronald Chump assumes the bank is covering this for him and skips this step.

Always pay \$10K for your traveling staff. Pay another \$10K for each open field office. If you run short of money, immediately close all field offices. This
means that Earnie Painter loses his free field office in New Hampshire, if he runs out of money.

\subsection{Play a Weapon}
Each weapon card has instructions. Follow them. Here are one oddity and some reminders. Only one scandal can be in effect against a player at a time.
Remember to keep Stump, Teach Your Candidate, and Endorsement cards. Only discard them if a scandal takes effect and tells you to. Keep these cards face up
where everyone can see them.

If you can't play a weapon, or simply don't want to, skip to opening and closing offices.

\subsection{Field Offices}
On each turn, you may open one field office, but no player may have no more than three offices open in any state.
You may also close as many fields offices as you like, to conserve money on future turns.

Place a field office piece on one of the states on the board to open it. Take pieces off the board to close them. Field offices improve the effect of a
successful stump speach, but they cost \$10K per turn. Plan accordingly.

If you are playing with two players, you may only open offices in Iowa. If three or four are playing, you may open offices in Iowa or New Hampshire. Add
South Carolina only if five are playing.

\subsection{Draw}
If you played a weapon card, draw one at the end of your turn. You should always start and end a turn with five cards. If you forget to draw, do so
anytime (even during another player's turn). If you did not play a card, you may elect to discard at the end of the turn and draw a replacement. Do not
worry about whether you were unable to play or chose not to. In either case you may discard and draw, or you may keep your hand as is.

\section{Voting Day}

When a player moves the calendar piece to Iowa Caucus day the game ends immediately. The player with the most poll chips wins. If there is a tie for first,
those players advance to the New Hampshire primary. All players not in the tie place their chips on undecided.
Move the calendar back two spaces for each player in the tie plus one. Continue play. Or, you could call it a draw, flip a coin, or settle in
some other non-violent manner.

\subsection{Disclaimer}

Any similarity between persons in this game and any person living or dead is purely coincidental, unless that person ran for office, hosted or appeared on a TV show
(cable or broadcast) in which case they are public figures and should expect a certain amount of humor at their expense. Oh, and I have only good things to say about Nancy Kassebaum.

\end{document}
