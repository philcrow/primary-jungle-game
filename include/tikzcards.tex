% This was taken originally from
% http://tex.stackexchange.com/questions/47924/creating-playing-cards-using-tikz/185830#185830
%
% Modified by Phil Crow to suit text only cards with fewer do-dads.

%   COMMANDS ZUM ZUSAMMENBAUEN DER KARTEN
%   ---------------------------------------

%   TikZ/PGF Settings für die Karten
\pgfmathsetmacro{\cardwidth}{6}
\pgfmathsetmacro{\cardheight}{9}
\pgfmathsetmacro{\imagewidth}{\cardwidth}
\pgfmathsetmacro{\imageheight}{0.75*\cardheight}
\pgfmathsetmacro{\stripwidth}{0.7}
\pgfmathsetmacro{\strippadding}{0.2}
\pgfmathsetmacro{\textpadding}{0.1}
\pgfmathsetmacro{\titley}{\cardheight-\strippadding-1.5*\textpadding-0.5*\stripwidth}


%   Formen der einzelnen Kartenelemente/-bestandteile
\def\shapeCard{(0,0) rectangle (\cardwidth,\cardheight)}
\def\shapeLeftStripLong{(\strippadding,-0.2) rectangle (\strippadding+\stripwidth,\cardheight-\strippadding-\strippadding-1)}
\def\shapeLeftStripShort{(\strippadding,\cardheight-\strippadding-1) rectangle (\strippadding+\stripwidth,\cardheight+0.2)}
\def\shapeRightStripShort{(\cardwidth-\stripwidth-\strippadding,\cardheight-\strippadding-1) rectangle (\cardwidth-\strippadding,\cardheight+0.2)}
\def\shapeTitleArea{(2*\strippadding-0.25*\stripwidth,\cardheight-\strippadding) rectangle (\cardwidth-2*\strippadding+0.25*\stripwidth,\cardheight-2*\stripwidth)}
\def\shapeContentArea{(2*\strippadding+\stripwidth,0.5*\cardheight) rectangle (\cardwidth+0.2,-0.2)}


%   Stylings für die Elemente definieren
\tikzset{
    %   runde Ecken für die Karten
    cardcorners/.style={
        rounded corners=0.2cm
    },
    %   runde Ecken für die "Fähnchen"
    elementcorners/.style={
        rounded corners=0.1cm
    },
    %   Bild für das Kartenmotiv
    cardimage/.style={
        path picture={
            \node[below=-1.5mm] at (0.5*\cardwidth,\cardheight) {
                \includegraphics[width=\imagewidth cm]{#1}
            };
        }
    },
}

%   TikZ-Raster
\newcommand{\carddebug}{
    \draw [step=1,help lines] (0,0) grid (\cardwidth,\cardheight);
}

%   Rahmen der Karte
\newcommand{\cardborder}{
    \draw[lightgray,cardcorners] \shapeCard;
}

%   Hintergrund der Karte
\newcommand{\cardbackground}[1]{
    \draw[cardcorners, cardimage=#1] \shapeCard;
}

%   Kategorie der Karte
\newcommand{\cardtype}[3]{
    %   First we fill the intersecting area
    %   The \clip command does not allow options, therefore 
    %   we have to use a scope to set the even odd rule.
    \begin{scope}[even odd rule]
        %   Define a clipping path. All paths outside shapeCard will
        %   be cut because the even odd rule is set.
        \clip[cardcorners] \shapeCard;
    \end{scope}

}
\newcommand{\cardtypeCharacter}{\cardtype{characterbg}{Charaktereigenschaft}{\hspace{-1mm}\LARGE\lefthand}}
\newcommand{\cardtypeAbility}{\cardtype{abilitybg}{Fähigkeit}{\hspace{-1mm}\Large\floweroneright}}
\newcommand{\cardtypeItem}{\cardtype{itembg}{Gegenstand}{\hspace{-1mm}\LARGE\bomb}}
\newcommand{\cardtypeTest}{\cardtype{testbg}{Testkarte}{\hspace{-1.4mm}\huge\ding{78}}}

%   Titel der Karte
\newcommand{\cardtitle}[1]{
    \node[text width=3.75cm] at (0.5*\cardwidth,\titley) {
        \begin{center}
            {\Large \textbf{#1}}
        \end{center}
    };
}

%   Inhalt der Karte
\newcommand{\cardcontent}[2]{
    \begin{scope}[even odd rule]
        \clip[cardcorners] \shapeCard;
        \fill[elementcorners,contentbg] \shapeContentArea;
    \end{scope}
    \node[below right, text width=(\cardwidth-2*\strippadding-2*\textpadding)*1cm] at (2*\strippadding-\textpadding,0.85*\cardheight-\textpadding) {
        %\textit{\glqq\normalsize #1\grqq}
        \textit{\normalsize #1}
    };
    \node[below right, text width=(\cardwidth-2*\strippadding-2*\textpadding-0.3)*1cm] at (2*\strippadding-\textpadding,5.8) {
        \vrule width \textwidth height 2pt \\[-2pt]
        \vspace{0.2cm}
        \textbf{\scriptsize #2}
    };
}

\newcommand{\candidatecontent}[2]{
    \begin{scope}[even odd rule]
        \clip[cardcorners] \shapeCard;
        \fill[elementcorners,contentbg] \shapeContentArea;
    \end{scope}
    \node[below] at (3,0.85*\cardheight-\textpadding) {
        \textit{\normalsize #1}
    };
    \node[below right, text width=(\cardwidth-2*\strippadding-2*\textpadding-0.3)*1cm] at (2*\strippadding-\textpadding,5.8) {
        \vrule width \textwidth height 2pt \\[-2pt]
        \vspace{0.2cm}
        \textbf{\scriptsize #2}
    };
}

%   Preis der Karte
\newcommand{\cardprice}[1]{
    \begin{scope}[even odd rule]
        \clip[cardcorners] \shapeCard;
    \end{scope}
    \node at (\cardwidth-0.5*\stripwidth-\strippadding,\titley-0.1) {\color{white}#1};
}
