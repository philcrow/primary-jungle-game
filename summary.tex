\documentclass[twocolumn]{article}
\begin{document}
\title{Primary Jungle Game Summary}
\author{Phil Crow}
\maketitle

\section{Premise}

Players are campaign managers who first choose a candidate. They guide the candidate to the Iowa Jungle Caucus in which all candidates compete together
regardless of party. Each candidate has poll chips which are equally divided at the start. Each turn moves the voting closer. When voting day arrives the
candidate with the most poll chips wins.

\section{Candidates}

There are 16 candidates to choose from. This is probably too many, but making them up was the most fun. Here are some examples:

Fiona Carlson     - Business executive hired a few, layed off a few more.

Ronald Chump      - Business magnate, I'm afraid to say more. He can be a tad litigious and I wrote the disclaimer myself.

Ed Coast          - Southern senator who loves Jesus and hates immigrants, except himself (he was born in Mexico, to a US mother and a Swedish father), never smiles.

Mallory Hinton    - Former senator and secretary of state who long ago stepped out of her political husband's long shadow. Still wants the validation of winning one more race.

Earnie Painter - Sitting senator from New England with a strong a affinity for cold winters, especially those in Scandanavia. Feel the Pain'.

Mick Sanatarium   - Former purple state senator, lost big, still trying for a comeback. Wishes google observed single issue right to be forgotten in the US like it does in Europe.

Jethro Shrub      - Fourth son of great old political family. Wonders why all these other people have bothered to run.

Aaron Snore - Former senator from a southern state who loves to talk about the problem of climate change when he is not running for office.

\subsection{Disclaimer}

Any similarity between persons in this game and any person living or dead is purely coincidental, unless that person ran for office, hosted or appeared on a TV show
(cable or broadcast) in which case they are public figures and should expect a certain amount of humor at their expense. Oh, and I have only good things to say about Nancy Kassebaum.

\section{Mechanics}

There is a board for the calendar, draw and discard piles, and states where candidates open field offices.

On each turn candidates raise money, pay expenses, play a weapon card, open or close offices, and draw.

The key activity is playing a weapon card from a hand of five. Each weapons has a category. Categories include: Attack Ad, Scandal, Defeat Scandal, Stump for a group, Endorsement.
There are 137 weapon cards, instead of 140, because honesty, integrity, and foresight don't seem apply to any of the candidates.

\section{Weapons}

Each weapon card is unique. The rules for play are the same for each card type. Currently the rules are printed on each card.

Here are some examples of the weapons:

\subsection{Attack Ads}

Claim your opponent supports the mohair subsidy.

Claim your opponent wants to take away everyone's steak knives.

Claim your opponent denies the theory of gravity.

Suggest that your opponent favors requiring each voter to recite the pledge of allegiance before getting a ballot.

Remind voters that your opponent still uses a flip phone.

Accuse your opponent of wanting to tax fingernail clippers.

Accuse your opponent of preferring soccer to North American football.

Accuse your opponent of wanting to tax Little League tickets.

Show your opponent wearing plaid Bermuda shorts, with black knee high socks and sandals.

Claim your opponent supports NSA surveillance devices embedded in TV remote controls.

\subsection{Endorsement from...}

the National Association of Puppy Owners

the Society of Underweight Engineers

Parents of Children with Excellent Teeth (PCET)

National Reptile Association (the real NRA)

Student Selfie Society

Fraternal Order of \textbf{Lawn} Enforcement Officers

Clown Car Owner's Association

United Tire Balancers Union

\subsection{Stump for a Group}

Stump speeches are more effective if given in states where the candidate has field offices open. Here are some groups canidates might stump for:

a high school chess club

a group of mall Santas

a pie making contest

a federation of foosball players

a rotary dial phone owner's club

a convention of county seat tourist bureau coordinators

a county fair rabbit show

the pessimists club national convention

a travelling unicycle troup tryout camp

a group of Johnny Cash impersonators

the contestants in a lawn tractor pull

\subsection{Teach Your Candidate}

There are various cards for instructing the candidates. These improve the effect of policy speeches. You could teach them:

the difference between a tax bracket and a tax loophole.

the relationship between the national debt and the current year budget deficit.

the difference between Slovenia and Slovakia.

the difference between Idaho and Iowa.

the difference between state and federal budgeting.

how to order at a fast food restaurant.

how much a typical worker actually makes.

that cloud computing is not affected by the jet stream.

\subsection{Speak}

Your candidate might choose to give a speech about one of these policy topics:

the rising price of envelopes

Estonian relations

the tax on suction cup tipped arrows

the need for a tax on dental floss

the risk to Thanksgiving due to the monopoly on canned pumpkin

the porousness of our Canadian border

the importance of uniforms for college students

the need for more skilled penny polishers

the beauty of the national parks your candidate has not visited

\end{document}
